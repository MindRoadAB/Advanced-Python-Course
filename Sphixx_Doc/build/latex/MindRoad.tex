%% Generated by Sphinx.
\def\sphinxdocclass{report}
\documentclass[letterpaper,10pt,english]{sphinxmanual}
\ifdefined\pdfpxdimen
   \let\sphinxpxdimen\pdfpxdimen\else\newdimen\sphinxpxdimen
\fi \sphinxpxdimen=.75bp\relax

\PassOptionsToPackage{warn}{textcomp}
\usepackage[utf8]{inputenc}
\ifdefined\DeclareUnicodeCharacter
% support both utf8 and utf8x syntaxes
\edef\sphinxdqmaybe{\ifdefined\DeclareUnicodeCharacterAsOptional\string"\fi}
  \DeclareUnicodeCharacter{\sphinxdqmaybe00A0}{\nobreakspace}
  \DeclareUnicodeCharacter{\sphinxdqmaybe2500}{\sphinxunichar{2500}}
  \DeclareUnicodeCharacter{\sphinxdqmaybe2502}{\sphinxunichar{2502}}
  \DeclareUnicodeCharacter{\sphinxdqmaybe2514}{\sphinxunichar{2514}}
  \DeclareUnicodeCharacter{\sphinxdqmaybe251C}{\sphinxunichar{251C}}
  \DeclareUnicodeCharacter{\sphinxdqmaybe2572}{\textbackslash}
\fi
\usepackage{cmap}
\usepackage[T1]{fontenc}
\usepackage{amsmath,amssymb,amstext}
\usepackage{babel}
\usepackage{times}
\usepackage[Bjarne]{fncychap}
\usepackage{sphinx}

\fvset{fontsize=\small}
\usepackage{geometry}

% Include hyperref last.
\usepackage{hyperref}
% Fix anchor placement for figures with captions.
\usepackage{hypcap}% it must be loaded after hyperref.
% Set up styles of URL: it should be placed after hyperref.
\urlstyle{same}

\addto\captionsenglish{\renewcommand{\figurename}{Fig.}}
\addto\captionsenglish{\renewcommand{\tablename}{Table}}
\addto\captionsenglish{\renewcommand{\literalblockname}{Listing}}

\addto\captionsenglish{\renewcommand{\literalblockcontinuedname}{continued from previous page}}
\addto\captionsenglish{\renewcommand{\literalblockcontinuesname}{continues on next page}}
\addto\captionsenglish{\renewcommand{\sphinxnonalphabeticalgroupname}{Non-alphabetical}}
\addto\captionsenglish{\renewcommand{\sphinxsymbolsname}{Symbols}}
\addto\captionsenglish{\renewcommand{\sphinxnumbersname}{Numbers}}

\addto\extrasenglish{\def\pageautorefname{page}}

\setcounter{tocdepth}{1}



\title{MindRoad Documentation}
\date{Oct 10, 2018}
\release{1.0}
\author{Tony Homsi}
\newcommand{\sphinxlogo}{\vbox{}}
\renewcommand{\releasename}{Release}
\makeindex
\begin{document}

\maketitle
\sphinxtableofcontents
\phantomsection\label{\detokenize{index::doc}}


Hello This is Tony from MindRoad AB

\noindent\sphinxincludegraphics[width=500\sphinxpxdimen,height=100\sphinxpxdimen]{{MindRoad2}.png}


\chapter{MindRoad Academy}
\label{\detokenize{test:mindroad-academy}}\label{\detokenize{test::doc}}

\section{Introduction}
\label{\detokenize{test:introduction}}
This is a sample tutorial to use Sphinxdoc from MindRoad Academy


\subsection{List of Departments}
\label{\detokenize{test:list-of-departments}}\begin{description}
\item[{\sphinxstyleemphasis{MindRoad AB} Contains the following departments:}] \leavevmode\begin{itemize}
\item {} 
\sphinxstylestrong{Academy} Group

\item {} 
\sphinxstylestrong{Embedded} Group

\item {} 
\sphinxstylestrong{Application} Group

\item {} 
\sphinxstylestrong{Sourcing}

\end{itemize}

\end{description}

\begin{sphinxadmonition}{note}{Note:}
Here you can add a note!
\end{sphinxadmonition}

\begin{sphinxadmonition}{warning}{Warning:}
Here you can add a warning message!
\end{sphinxadmonition}
\begin{itemize}
\item {} 
To Write a Code

\end{itemize}

\fvset{hllines={, ,}}%
\begin{sphinxVerbatim}[commandchars=\\\{\}]
\PYG{n}{Print}\PYG{p}{(}\PYG{l+s+s2}{\PYGZdq{}}\PYG{l+s+s2}{Hello from MindRoad}\PYG{l+s+s2}{\PYGZdq{}}\PYG{p}{)}
\end{sphinxVerbatim}
\begin{itemize}
\item {} 
This is a paragraph contains a link \sphinxhref{http://www.mindroad.se/sv-SE}{MindRoad AB} website .

\end{itemize}
\begin{itemize}
\item {} 
To write a Mathematical equation:

\end{itemize}
\begin{equation*}
\begin{split}n_{\mathrm{offset}} = \sum_{k=1}^{N-10} a_k b_k\end{split}
\end{equation*}\begin{itemize}
\item {} 
You can Create a Table as following:

\end{itemize}


\begin{savenotes}\sphinxattablestart
\centering
\begin{tabular}[t]{|*{3}{\X{1}{3}|}}
\hline
\sphinxstyletheadfamily 
H 1
&\sphinxstyletheadfamily 
H 2
&\sphinxstyletheadfamily 
H 3
\\
\hline
row 1
&
column 2
&
column 3
\\
\hline
row 2
&\sphinxstartmulticolumn{2}%
\begin{varwidth}[t]{\sphinxcolwidth{2}{3}}
Cells may span columns.
\par
\vskip-\baselineskip\vbox{\hbox{\strut}}\end{varwidth}%
\sphinxstopmulticolumn
\\
\hline
row 3
&\sphinxmultirow{2}{10}{%
\begin{varwidth}[t]{\sphinxcolwidth{1}{3}}
Cells may
span rows.
\par
\vskip-\baselineskip\vbox{\hbox{\strut}}\end{varwidth}%
}%
&\sphinxmultirow{2}{11}{%
\begin{varwidth}[t]{\sphinxcolwidth{1}{3}}
\begin{itemize}
\item {} 
Cells

\item {} 
contain

\item {} 
blocks.

\end{itemize}
\par
\vskip-\baselineskip\vbox{\hbox{\strut}}\end{varwidth}%
}%
\\
\cline{1-1}
row 4
&\sphinxtablestrut{10}&\sphinxtablestrut{11}\\
\hline
\end{tabular}
\par
\sphinxattableend\end{savenotes}

\begin{center}Create a centered boldfaced text: Python version 3.7
\end{center}\begin{itemize}
\item {} 
To implement emphasis:

\end{itemize}

\sphinxstyleemphasis{MindRoad Academy erbjuder spetsutbildningar inom data- och telekom, inbyggda system om realtidsutveckling, applikationstuveckling samt agila metoder.}
\begin{itemize}
\item {} 
To download a file

\end{itemize}

\sphinxcode{\sphinxupquote{download test.html}}


\chapter{Indices and tables}
\label{\detokenize{index:indices-and-tables}}\begin{itemize}
\item {} 
\DUrole{xref,std,std-ref}{genindex}

\item {} 
\DUrole{xref,std,std-ref}{modindex}

\item {} 
\DUrole{xref,std,std-ref}{search}

\end{itemize}



\renewcommand{\indexname}{Index}
\printindex
\end{document}